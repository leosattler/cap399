%%%%%%%%%%%%%%%%%%%%%%%%%%%%%%%%%%%%%%%%%%%%%%%%%%%%%%%%%%%%%%%%%%%%%%%%%%%%%%%%
% RESUMO %% obrigatório

\begin{resumo}

No presente trabalho, os compiladores \texttt{Intel v.19.1}, \texttt{PGI v.19.4} e \texttt{GNU v.8.3} são avaliados quanto à capacidade de vetorização. Foi utilizada uma versão ligeiramente modificada da coleção de testes TSVC (Test Suite for Vectorizing Compilers) descrita em \citeonline{maleki2011evaluation}. Este manuscrito está dividido em três seções: na Seção I, os resultados da bateria de testes são apresentados, caracterizando os ganhos e desempenhos obtidos pela vetorização de cada compilador; na Seção II, as flags de compilação são alteradas de modo a buscar \textit{Speedups} superiores aos obtido durante os testes da primeira seção; as conclusões são apresentadas na Seção III.

%% neste arquivo resumo.tex
%% o texto do resumo e as palavras-chave têm que ser em Português para os documentos escritos em Português
%% o texto do resumo e as palavras-chave têm que ser em Inglês para os documentos escritos em Inglês
%% os nomes dos comandos \begin{resumo}, \end{resumo}, \palavraschave e \palavrachave não devem ser alterados

\hypertarget{estilo:resumo}{} %% uso para este Guia

%\palavraschave{%
%	\palavrachave{Wavelet analysis}%
%	\palavrachave{Gapped data}%
%	\palavrachave{Numerical methods}%
%	\palavrachave{Lomb-Scargle periodogram}%
%	\palavrachave{Spectral analysis}%
%}
 
\end{resumo}